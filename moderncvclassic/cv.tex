%% Based on file `template.tex'.
%% Copyright 2006-2013 Xavier Danaux (xdanaux@gmail.com).
%
% This work may be distributed and/or modified under the
% conditions of the LaTeX Project Public License version 1.3c,
% available at http://www.latex-project.org/lppl/.


\documentclass[11pt,a4paper,sans]{moderncv}
% possible options include font size ('10pt', '11pt' and '12pt'), paper size ('a4paper', 'letterpaper', 'a5paper', 'legalpaper', 'executivepaper' and 'landscape') and font family ('sans' and 'roman')

\moderncvstyle{classic} % style options are 'casual' (default), 'classic', 'oldstyle' and 'banking'
\moderncvcolor{green} % color options 'blue' (default), 'orange', 'green', 'red', 'purple', 'grey' and 'black'

%\providecommand{\tightlist}{\setlength{\itemsep}{0pt}\setlength{\parskip}{0pt}}
\providecommand{\tightlist}{}
%\renewcommand{\familydefault}{\sfdefault} % to set the default font; use '\sfdefault' for the default sans serif font, '\rmdefault' for the default roman one, or any tex font name

\usepackage[utf8]{inputenc}

% adjust the page margins
\usepackage[scale=0.75]{geometry}
%\setlength{\hintscolumnwidth}{3cm}                % if you want to change the width of the column with the dates
%\setlength{\makecvtitlenamewidth}{10cm}           % for the 'classic' style, if you want to force the width allocated to your name and avoid line breaks. be careful though, the length is normally calculated to avoid any overlap with your personal info; use this at your own typographical risks...

% personal data
\name{Leandro Lisboa Penz}{}
%\title{Resumé title}                               % optional, remove / comment the line if not wanted
%\address{street}{zip, city}{country}
\email{lpenz@lpenz.org}

\homepage{http://www.lpenz.org}
\social[linkedin]{lpenz}
\social[github]{lpenz}
\social[twitter][www.twitter.com/lpenz]{lpenz}

%\extrainfo{additional information}
%\photo[64pt][0.4pt]{lpenz.png}
%\quote{Some quote}

\begin{document}

\makecvtitle

\section{About me}

I'm currently working as an embedded software architect of DmOS ---
DATACOM's universal embedded OS. I focus on modularity, continuous
integration, incremental qualification and DevOps. I work very closely
with the DevOps team, bringing solutions and new technologies, and also
helping in general.

\vspace{1em}
I was one of the architects that worked on the initial definitions of
the architecture, processes and tools used nowadays in the company. The
architecture group itself was something new back then.

\vspace{1em}
Before the creation of the architecture group, I led small (8)
development teams. At first, I led teams of embedded software
development for about 5 years, as I started as a developer myself. I
then led the test infrastructure team for about a year while also acting
as a software architect for the Ethernet switch product line.

\vspace{1em}
I hold a master degree in computer science from UFRGS. My research line
was on computer networks, distributed systems, fault tolerance and
security. My dissertation was about distributed firewall rule coherence.
I implemented the prototype checker in Haskell.

\vspace{1em}
I hold an MBA on management and leadership from UNISINOS. I graduated in
electrical engineering from UFRGS.

\vspace{1em}
I value technical excellence, out-of-the-box thinking, generalism,
strategic planning and long-term maintainability. I'm always looking for
vulnerabilities and enhancement opportunities in my knowledge and tool
set.

\vspace{1em}
Out of the professional context, I have studied and obtained a license
to fly sailplanes. I can definitely say that I learned a lot from doing
that, from the differences with my area (extreme risk aversion) as well
as from the similarities (briefing/debriefing \(\sim\) sprint
planning/review).


\clearpage

\section{Experience}


\cventry{2016 - present}{Software engineer}{Arista Networks}{}{}{}

\cventry{2014 - 2016}{Embedded software architect}{DATACOM}{}{}{Software architect of DATACOM's embedded universal operating system: %
DmOS. %
\begin{itemize} %
\tightlist %
\item %
  Research and high-level definition the system architecture. %
\item %
  Design of the overall development process. %
\item %
  A lot of code review. %
\item %
  Coding of some core modules. %
\end{itemize} %
 %
Software architect assigned to DevOps. %
\begin{itemize} %
\tightlist %
\item %
  Initial tooling and structuring of the DevOps infrastructure. %
\item %
  Definition of the DevOps process: homologation vs test vs production %
  environments. %
\item %
  Alignment of the infrastructure with DmOS's requirements. %
\item %
  Coding of some of the solutions. %
\end{itemize} %
}

\cventry{2012 - 2014}{Leader of test automation team; embedded software architect}{DATACOM}{}{}{Software Team Leader of Ethernet switch test automation team. %
\begin{itemize} %
\tightlist %
\item %
  Coordination of the team that was responsible for: %
\begin{itemize} %
  \tightlist %
  \item %
    test development: in ruby; %
  \item %
    test excution: 24/7, in sync with product integration; %
  \item %
    test infrastructure maintenance: \(\sim\) 5 testbeds, most with %
    Ethernet ring topologies. %
  \end{itemize} %
\item %
  Backlog and defect priorization. %
\item %
  Agile methodologies. %
\item %
  Performance evaluation, feedback. %
\end{itemize} %
 %
Software architect of DATACOM's Ethernet switch product line: %
\begin{itemize} %
\tightlist %
\item %
  Evaluation and implementation of static analysis tools. %
\item %
  Development of debugging tools. %
\item %
  Overall development process enhancements. %
\item %
  I was also heavily involed with the design and implementation of tools %
  that gathered data and generated development and defect metrics using %
  python and django, full stack. %
\end{itemize} %
}

\cventry{2007 - 2011}{Leader of embedded software team}{DATACOM}{}{}{Software team leader of DATACOM's NG-SDH Multiplexer product line. %
\begin{itemize} %
\tightlist %
\item %
  Coordination of distributed teams (\(\sim\) 8, about half local, half %
  remote) of embedded software developers. %
\item %
  Guidance and reference in the development of solutions. %
\item %
  Backlog and defect priorization. %
\item %
  Agile methodologies. %
\item %
  Performance evaluation, feedback. %
\item %
  A lot of code review. %
\end{itemize} %
}

\cventry{2002 - 2007}{Embedded software developer}{DATACOM}{}{}{Team member of the first NG-SDH Multiplexer developed in Brazil. %
\begin{itemize} %
\tightlist %
\item %
  Software development using the C linguage for embedded Linux systems: %
  network servers, debug tools and drivers (kernel space). %
\item %
  Unix server administration and maintenance. %
\item %
  Script development (test automation) and scripting language extension: %
  Lua, Python and TCL (Expect). %
\end{itemize} %
}


\clearpage

\section{Education}


\cventry{2012 - 2013}{MBA - Organizational Leadership and Management}{Vale do Rio dos Sinos University (UNISINOS)}{}{}{Subjects: %
\begin{itemize} %
\tightlist %
\item %
  Business management %
\item %
  Project management %
\item %
  Human resources %
\item %
  Leadership %
\item %
  Organizational identity and culture %
\end{itemize} %
}

\cventry{2006 - 2008}{Master of Science, Computer Science}{Federal University of Rio Grande do Sul (UFRGS)}{}{}{\emph{Orientation}: Prof.~Dr.~Raul Fernando Weber\\ %
\emph{Research line}: fault tolerance, network security and distributed %
systems.\\ %
\emph{Dissertation}: Coherence in distributed packet filters\\ %
Formal definition of the concept of anomaly in isolated and distributed %
network packet filters through the use of graphs and set theory. %
Development of a prototype anomaly checker in Haskell.\\ %
\emph{Online}: \url{http://www.lume.ufrgs.br/handle/10183/22813} %
}

\cventry{1998 - 2002}{Graduation, Electrical Engineering}{Federal University of Rio Grande do Sul (UFRGS)}{}{}{The program is ranked among the top in Brazil, according to the %
evaluation process carried out by the CAPES agency of the Brazilian %
Ministry of Education. %
}



\section{Other courses}


\cventry{2009}{Agile methodologies: project management with SCRUM}{Pontifícia Universidade Católica - RS}{}{Course load: 12h}{}

\cventry{2006}{Software project management}{Vale do Rio dos Sinos University (UNISINOS)}{}{Course load: 21h}{Study of PMI's PMBOK. %
}

\cventry{2003}{TCP/IPv4 routing - Internet high availability}{UFRGS - Data processing center}{}{Course load: 16h}{Focused on network routing protocols: RIP, OSPF and BGP. %
}



\section{Languages}


\cvitemwithcomment{Portuguese}{Native}{}

\cvitemwithcomment{English}{Fluent}{}

\cvitemwithcomment{German}{Basic}{}


\clearpage

\section{Random achievements and anecdotes}

\cvlistitem{I installed my first Debian GNU/Linux at home in 2003, and never
reinstalled it from scratch. I use the ``testing'' release, and update
it constantly.
}

\cvlistitem{I learned Haskell while developing the prototype of my MSc dissertation
in 2006. Since then, I have used it on side projects and on programming
contests when I have the chance.
}

\cvlistitem{Even though I did not take computer science at graduation, I was 6th on
the entry exam of the MSc at my University (UFRGS).
}

\cvlistitem{I once got all questions right in a logical reasoning test. The
psychologist that evaluated it said that she had never seen that before.
}

\cvlistitem{This CV is maintained at \url{http://github.com/lpenz/cv} with
configuration management, automated tests and continuous deployment to
\url{http://cv.lpenz.org} in several output formats.
}



\end{document}

% vim: ft=tex.jinja